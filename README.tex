% Intended LaTeX compiler: pdflatex
\documentclass[french]{report}
\usepackage[utf8]{inputenc}
\usepackage[T1]{fontenc}
\usepackage{graphicx}
\usepackage{grffile}
\usepackage{longtable}
\usepackage{wrapfig}
\usepackage{rotating}
\usepackage[normalem]{ulem}
\usepackage{amsmath}
\usepackage{textcomp}
\usepackage{amssymb}
\usepackage{capt-of}
\usepackage{hyperref}
\usepackage{listings}
\usepackage{xcolor}
\usepackage[french]{babel}
%% fichier-config-listings.tex -- LaTeX2e My common commands and environments
%
% Copyright (C) 2008-2021 Fabrice Niessen. All rights reserved.

%* --[ Listings --------------------------------------------------------------

\definecolor{mc@lstidentifier}{HTML}{000000} % black
\definecolor{mc@lstcomment}{HTML}{008200} % green
\definecolor{mc@lststring}{HTML}{FF5500} % orange
\definecolor{mc@lstkeyword}{HTML}{0000FF} % blue
\definecolor{mc@lstbackground}{HTML}{FFFFCC} % light yellow
\definecolor{mc@lstframe}{HTML}{FFEE88} % dark yellow

\lstdefinelanguage{org}{%
  morekeywords={:results, :session, :var, :noweb, :exports},
  sensitive=false,
  morestring=[b]",
  morecomment=[l]{\#},
}

% a new language must be defined before being loaded: hence, put
% `lstloadlanguages' after having defined languages that should be loaded
\lstloadlanguages{% check documentation for other languages...
  C,
  %C++,
  %XML,
  %HTML,
  Java,
  Python,
  R,
  sword
}

\lstset{%
  lineskip=-0.1em,
  %
  basicstyle=\ttfamily\scriptsize, % font that is used for the code
  identifierstyle=\color{mc@lstidentifier},
  commentstyle=\color{mc@lstcomment}\itshape,
  stringstyle=\color{mc@lststring},
  keywordstyle=\color{mc@lstkeyword},
  %
  extendedchars=true,
  inputencoding=utf8,
  upquote, %
  tabsize=4, % set default tabsize to 4 spaces
  showtabs=false, % show tabs within strings adding particular underscores
  %  tab=$\to$,
  showspaces=false, % show spaces adding particular underscores
  showstringspaces=false, % underline spaces within strings
  %
  numbers=left, % where to put the line numbers
  stepnumber=0, % step between two line numbers
  numberstyle=\tiny, % line number font size
  %
  captionpos=b, % set the caption position to `bottom'
  %
  xleftmargin=0.4em, % text to the right
  xrightmargin=0.4em, % text to the left
  breaklines=false, % don't break long lines of code
  %
  frame=single, % add a frame around the code
  framexleftmargin=0pt, % frame back to the left
  framexrightmargin=0pt, % frame back to the right
  backgroundcolor=\color{mc@lstbackground}, % set the background color
  rulecolor=\color{mc@lstframe}, % frame color
  %
  columns=flexible, % try not to ruin the spacing intended by the font designer
  keepspaces=true, % don't drop spaces to fix column alignment
  %
  % mathescape, % allow escaping to (La)TeX mode within $..$
  escapechar=², % allow escaping to (La)TeX mode within ²..²
  % The backquote was NOT judicious: in some code (comments), I wrap vars
  % between such a backquote (`var')
  %
  % conversion of UTF-8 chars to latin1
  literate=
  { }{{~}}1
  {á}{{\'a}}1
  {à}{{\`a}}1
  {â}{{\^a}}1
  {ä}{{\"a}}1
  {ç}{{\c{c}}}1
  {é}{{\'e}}1
  {è}{{\`e}}1
  {ê}{{\^e}}1
  {ë}{{\"e}}1
  {í}{{\'i}}1
  {ì}{{\`i}}1
  {î}{{\^i}}1
  {ï}{{\"i}}1
  {ó}{{\'o}}1
  {ò}{{\`o}}1
  {ô}{{\^o}}1
  {ö}{{\"o}}1
  {ú}{{\'u}}1
  {ù}{{\`u}}1
  {û}{{\^u}}1
  {ü}{{\"u}}1
  {Á}{{\'A}}1
  {À}{{\`A}}1
  {Â}{{\^A}}1
  {Ä}{{\"A}}1
  {Ç}{{\c{C}}}1
  {É}{{\'E}}1
  {È}{{\`E}}1
  {Ê}{{\^E}}1
  {Ë}{{\"E}}1
  {Í}{{\'I}}1
  {Ì}{{\`I}}1
  {Î}{{\^I}}1
  {Ï}{{\"I}}1
  {Ó}{{\'O}}1
  {Ò}{{\`O}}1
  {Ô}{{\^O}}1
  {Ö}{{\"O}}1
  {Ú}{{\'U}}1
  {Ù}{{\`U}}1
  {Û}{{\^U}}1
  {Ü}{{\"U}}1
}

% This is for the sake of Emacs.
% Local Variables:
% ispell-local-dictionary: "american"
% End:

%% fichier-config-listings.tex ends here

\author{Fabrice Niessen}
\date{\today}
\title{Literate programming avec Org mode}
\hypersetup{
 pdfauthor={Fabrice Niessen},
 pdftitle={Literate programming avec Org mode},
 pdfkeywords={gutenberg, emacs, org-mode, latex, booster},
 pdfsubject={Fichier de démo pour la conférence GUTenberg 2021},
 pdfcreator={Emacs 27.1 (Org mode 9.3)}, 
 pdflang={French}}
\begin{document}

\maketitle
Ceci est un document Org.

\textbf{Org mode} est (entre autres) une syntaxe \emph{plain text} facile à utiliser pour
rédiger des documents \LaTeX{} documents, créer des pages Web et encore plein
d'autres choses !

\textbf{Org mode} is a easy-to-write \emph{plain text} formatting syntax for authoring \LaTeX{}
documents, creating Web pages and much more!

\url{file:///cygdrive/d/Users/fni/tests/org-mode-syntax-example.org}

\chapter{Introduction}
\label{sec:org0980ec2}

Excogitatum est super his, ut homines quidam ignoti, vilitate ipsa parum cavendi
ad colligendos rumores per Antiochiae latera cuncta destinarentur relaturi quae
audirent. hi peragranter et dissimulanter honoratorum circulis adsistendo
pervadendoque divites domus egentium habitu quicquid noscere poterant vel audire
latenter intromissi per posticas in regiam nuntiabant, id observantes
conspiratione concordi, ut fingerent quaedam et cognita duplicarent in peius,
laudes vero supprimerent Caesaris, quas invitis conpluribus formido malorum
inpendentium exprimebat.

\chapter{Équipements et tests}
\label{sec:org0c6e83b}

Post haec indumentum regale quaerebatur et ministris fucandae purpurae tortis
confessisque pectoralem tuniculam sine manicis textam, Maras nomine quidam
inductus est ut appellant Christiani diaconus.

\section{Équipements utilisés}
\label{sec:org76a1d28}

Superatis Tauri montis verticibus qui ad solis ortum sublimius attolluntur,
Cilicia spatiis porrigitur late distentis dives bonis omnibus terra, eiusque
lateri dextro adnexa Isauria, pari sorte uberi palmite viget et frugibus
minutis, quam mediam navigabile flumen Calycadnus interscindit :

\begin{itemize}
\item ibi victu recreati et quiete,
\item postquam abierat timor,
\item vicos opulentos adorti equestrium adventu cohortium,
\item quae casu propinquabant.
\end{itemize}

\section{Conditions de tests}
\label{sec:orgc02e45d}

Itaque tum Scaevola cum in eam ipsam mentionem incidisset, exposuit nobis
sermonem Laeli de amicitia habitum ab illo secum et cum altero genero, C. Fannio
Marci filio, paucis diebus post mortem Africani :

\begin{enumerate}
\item inter quos Paulus eminebat notarius ortus in Hispania,
\item glabro quidam sub vultu latens,
\item odorandi vias periculorum occultas perquam sagax.
\end{enumerate}

\chapter{Résultats et discussion}
\label{sec:org0243216}

Hae duae provinciae bello quondam piratico catervis mixtae praedonum a Servilio
pro consule missae sub iugum factae sunt vectigales.

\begin{table}[!htbp]
\label{tab:org0c46daf}
\centering
\begin{tabular}{lr}
Jour & Frais\\
\hline
Lundi & 8\\
Mardi & 5\\
Mercredi & 2\\
Jeudi & 2\\
Vendredi & 3\\
Samedi & 1\\
Dimanche & 0\\
\hline
Total & 21\\
\end{tabular}
\end{table}

Quid?

\section{Résultats}
\label{sec:org45d157b}

Coactique aliquotiens nostri pedites ad eos persequendos scandere clivos
sublimes etiam si lapsantibus plantis fruticeta prensando vel dumos ad vertices
venerint summos.

\section{Discussion}
\label{sec:org7c64c77}

Sed tamen haec cum ita tutius observentur, quidam vigore artuum inminuto rogati
ad nuptias ubi aurum dextris manibus cavatis offertur, inpigre vel usque
Spoletium pergunt.

\chapter{Conclusions}
\label{sec:org7776d8a}

Inter haec Orfitus praefecti potestate regebat urbem aeternam ultra modum
delatae dignitatis sese efferens insolenter, vir quidem prudens et forensium
negotiorum oppido gnarus, sed splendore liberalium doctrinarum minus quam
nobilem decuerat institutus, quo administrante seditiones sunt concitatae graves
ob inopiam vini:

\begin{center}
\begin{tabular}{lrl}
\emph{Sous-total} & 21.00 & \texteuro{}\\
\hline
\emph{Taxe @ 21\%} & 4.41 & \texteuro{}\\
\hline
\textbf{Montant total} & \textbf{\large{25.41}} & \textbf{\texteuro{}}\\
\end{tabular}
\end{center}

Huius avidis usibus vulgus intentum ad motus asperos excitatur et crebros.

\chapter{Annexes}
\label{sec:orga31b3e4}

Code utilisé pour améliorer le rendu de la démonstration :

\lstset{language=Lisp,label= ,caption= ,captionpos=b,numbers=none}
\begin{lstlisting}
(set-frame-font "Consolas-11" nil t)
\end{lstlisting}

\lstset{language=Lisp,label= ,caption= ,captionpos=b,numbers=none}
\begin{lstlisting}
(setq org-latex-default-packages-alist
      '(("AUTO" "inputenc" t
         ("pdflatex"))
        ("T1" "fontenc" t
         ("pdflatex"))
        (#1="" "graphicx" t)
        ("" "xcolor")
        ("" "babel")
        ("" "listings")
        (#1# "hyperref" nil))
\end{lstlisting}

\lstset{language=Lisp,label= ,caption= ,captionpos=b,numbers=none}
\begin{lstlisting}
(setq org-latex-classes

      '(("article" "\\documentclass[11pt]{article}"
         ("\\section{%s}" . "\\section*{%s}")
         ("\\subsection{%s}" . "\\subsection*{%s}")
         ("\\subsubsection{%s}" . "\\subsubsection*{%s}")
         ("\\paragraph{%s}" . "\\paragraph*{%s}")
         ("\\subparagraph{%s}" . "\\subparagraph*{%s}"))

        ("report" "\\documentclass[11pt]{report}"
         ("\\chapter{%s}" . "\\chapter*{%s}")
         ("\\section{%s}" . "\\section*{%s}")
         ("\\subsection{%s}" . "\\subsection*{%s}")
         ("\\subsubsection{%s}" . "\\subsubsection*{%s}")
         ("\\paragraph{%s}" . "\\paragraph*{%s}"))

        ("book" "\\documentclass[11pt]{book}"
         ("\\chapter{%s}" . "\\chapter*{%s}")
         ("\\section{%s}" . "\\section*{%s}")
         ("\\subsection{%s}" . "\\subsection*{%s}")
         ("\\subsubsection{%s}" . "\\subsubsection*{%s}")
         ("\\paragraph{%s}" . "\\paragraph*{%s}"))))
\end{lstlisting}

\chapter{Suis-moi !}
\label{sec:orgcd75b43}

\begin{itemize}
\item Rejoins le groupe Facebook \\
et la communauté des « \textbf{Boosters Emacs} » ici : \\
\url{https://www.facebook.com/groups/286810496172489}

\item Abonne-toi à la chaîne Youtube « \textbf{Booster Emacs} » \\
\url{https://www.youtube.com/channel/UChSzlv1RJWFbb7XkV7OREtA?sub\_confirmation=1} \\
et active les notifications en cliquant sur la petite cloche pour ne rien
rater !
\end{itemize}
\end{document}
